\documentclass{article}
\usepackage[utf8]{inputenc}
\usepackage[brazilian]{babel}
\usepackage{subfigure}

\title{Sistema de arquivos}
\author{João Vitor, Millas Násser, Welton Santos}
\date{Dezembro 2017}

\usepackage{natbib}
\usepackage{graphicx}

\begin{document}

\maketitle

\section{Introdução}
	O trabalho apresenta uma simulação de um sistema de arquivo simples que é baseado em uma tabela de entradas. Utilizando de estruturas especiais que simulam a memória principal e um arquivo simulando o disco, além de tipos de dados que implementam diretórios e arquivos, cada entrada da tabela possui 16 bits e  armazena o endereço dos blocos de arquivos. 
	
	O objetivo principal é realizar a simulação e verificar a granularidade do disco ao final de uma série de comandos, a fim de verificar a fragmentação gerada.
    
\section{Conclusão}
    
\end{document}
